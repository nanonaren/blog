\documentclass{beamer}
%% ODER: format ==         = "\mathrel{==}"
%% ODER: format /=         = "\neq "
%
%
\makeatletter
\@ifundefined{lhs2tex.lhs2tex.sty.read}%
  {\@namedef{lhs2tex.lhs2tex.sty.read}{}%
   \newcommand\SkipToFmtEnd{}%
   \newcommand\EndFmtInput{}%
   \long\def\SkipToFmtEnd#1\EndFmtInput{}%
  }\SkipToFmtEnd

\newcommand\ReadOnlyOnce[1]{\@ifundefined{#1}{\@namedef{#1}{}}\SkipToFmtEnd}
\usepackage{amstext}
\usepackage{amssymb}
\usepackage{stmaryrd}
\DeclareFontFamily{OT1}{cmtex}{}
\DeclareFontShape{OT1}{cmtex}{m}{n}
  {<5><6><7><8>cmtex8
   <9>cmtex9
   <10><10.95><12><14.4><17.28><20.74><24.88>cmtex10}{}
\DeclareFontShape{OT1}{cmtex}{m}{it}
  {<-> ssub * cmtt/m/it}{}
\newcommand{\texfamily}{\fontfamily{cmtex}\selectfont}
\DeclareFontShape{OT1}{cmtt}{bx}{n}
  {<5><6><7><8>cmtt8
   <9>cmbtt9
   <10><10.95><12><14.4><17.28><20.74><24.88>cmbtt10}{}
\DeclareFontShape{OT1}{cmtex}{bx}{n}
  {<-> ssub * cmtt/bx/n}{}
\newcommand{\tex}[1]{\text{\texfamily#1}}	% NEU

\newcommand{\Sp}{\hskip.33334em\relax}


\newcommand{\Conid}[1]{\mathit{#1}}
\newcommand{\Varid}[1]{\mathit{#1}}
\newcommand{\anonymous}{\kern0.06em \vbox{\hrule\@width.5em}}
\newcommand{\plus}{\mathbin{+\!\!\!+}}
\newcommand{\bind}{\mathbin{>\!\!\!>\mkern-6.7mu=}}
\newcommand{\rbind}{\mathbin{=\mkern-6.7mu<\!\!\!<}}% suggested by Neil Mitchell
\newcommand{\sequ}{\mathbin{>\!\!\!>}}
\renewcommand{\leq}{\leqslant}
\renewcommand{\geq}{\geqslant}
\usepackage{polytable}

%mathindent has to be defined
\@ifundefined{mathindent}%
  {\newdimen\mathindent\mathindent\leftmargini}%
  {}%

\def\resethooks{%
  \global\let\SaveRestoreHook\empty
  \global\let\ColumnHook\empty}
\newcommand*{\savecolumns}[1][default]%
  {\g@addto@macro\SaveRestoreHook{\savecolumns[#1]}}
\newcommand*{\restorecolumns}[1][default]%
  {\g@addto@macro\SaveRestoreHook{\restorecolumns[#1]}}
\newcommand*{\aligncolumn}[2]%
  {\g@addto@macro\ColumnHook{\column{#1}{#2}}}

\resethooks

\newcommand{\onelinecommentchars}{\quad-{}- }
\newcommand{\commentbeginchars}{\enskip\{-}
\newcommand{\commentendchars}{-\}\enskip}

\newcommand{\visiblecomments}{%
  \let\onelinecomment=\onelinecommentchars
  \let\commentbegin=\commentbeginchars
  \let\commentend=\commentendchars}

\newcommand{\invisiblecomments}{%
  \let\onelinecomment=\empty
  \let\commentbegin=\empty
  \let\commentend=\empty}

\visiblecomments

\newlength{\blanklineskip}
\setlength{\blanklineskip}{0.66084ex}

\newcommand{\hsindent}[1]{\quad}% default is fixed indentation
\let\hspre\empty
\let\hspost\empty
\newcommand{\NB}{\textbf{NB}}
\newcommand{\Todo}[1]{$\langle$\textbf{To do:}~#1$\rangle$}

\EndFmtInput
\makeatother
%
%
%
%
%
%
% This package provides two environments suitable to take the place
% of hscode, called "plainhscode" and "arrayhscode". 
%
% The plain environment surrounds each code block by vertical space,
% and it uses \abovedisplayskip and \belowdisplayskip to get spacing
% similar to formulas. Note that if these dimensions are changed,
% the spacing around displayed math formulas changes as well.
% All code is indented using \leftskip.
%
% Changed 19.08.2004 to reflect changes in colorcode. Should work with
% CodeGroup.sty.
%
\ReadOnlyOnce{polycode.fmt}%
\makeatletter

\newcommand{\hsnewpar}[1]%
  {{\parskip=0pt\parindent=0pt\par\vskip #1\noindent}}

% can be used, for instance, to redefine the code size, by setting the
% command to \small or something alike
\newcommand{\hscodestyle}{}

% The command \sethscode can be used to switch the code formatting
% behaviour by mapping the hscode environment in the subst directive
% to a new LaTeX environment.

\newcommand{\sethscode}[1]%
  {\expandafter\let\expandafter\hscode\csname #1\endcsname
   \expandafter\let\expandafter\endhscode\csname end#1\endcsname}

% "compatibility" mode restores the non-polycode.fmt layout.

\newenvironment{compathscode}%
  {\par\noindent
   \advance\leftskip\mathindent
   \hscodestyle
   \let\\=\@normalcr
   \let\hspre\(\let\hspost\)%
   \pboxed}%
  {\endpboxed\)%
   \par\noindent
   \ignorespacesafterend}

\newcommand{\compaths}{\sethscode{compathscode}}

% "plain" mode is the proposed default.
% It should now work with \centering.
% This required some changes. The old version
% is still available for reference as oldplainhscode.

\newenvironment{plainhscode}%
  {\hsnewpar\abovedisplayskip
   \advance\leftskip\mathindent
   \hscodestyle
   \let\hspre\(\let\hspost\)%
   \pboxed}%
  {\endpboxed%
   \hsnewpar\belowdisplayskip
   \ignorespacesafterend}

\newenvironment{oldplainhscode}%
  {\hsnewpar\abovedisplayskip
   \advance\leftskip\mathindent
   \hscodestyle
   \let\\=\@normalcr
   \(\pboxed}%
  {\endpboxed\)%
   \hsnewpar\belowdisplayskip
   \ignorespacesafterend}

% Here, we make plainhscode the default environment.

\newcommand{\plainhs}{\sethscode{plainhscode}}
\newcommand{\oldplainhs}{\sethscode{oldplainhscode}}
\plainhs

% The arrayhscode is like plain, but makes use of polytable's
% parray environment which disallows page breaks in code blocks.

\newenvironment{arrayhscode}%
  {\hsnewpar\abovedisplayskip
   \advance\leftskip\mathindent
   \hscodestyle
   \let\\=\@normalcr
   \(\parray}%
  {\endparray\)%
   \hsnewpar\belowdisplayskip
   \ignorespacesafterend}

\newcommand{\arrayhs}{\sethscode{arrayhscode}}

% The mathhscode environment also makes use of polytable's parray 
% environment. It is supposed to be used only inside math mode 
% (I used it to typeset the type rules in my thesis).

\newenvironment{mathhscode}%
  {\parray}{\endparray}

\newcommand{\mathhs}{\sethscode{mathhscode}}

% texths is similar to mathhs, but works in text mode.

\newenvironment{texthscode}%
  {\(\parray}{\endparray\)}

\newcommand{\texths}{\sethscode{texthscode}}

% The framed environment places code in a framed box.

\def\codeframewidth{\arrayrulewidth}
\RequirePackage{calc}

\newenvironment{framedhscode}%
  {\parskip=\abovedisplayskip\par\noindent
   \hscodestyle
   \arrayrulewidth=\codeframewidth
   \tabular{@{}|p{\linewidth-2\arraycolsep-2\arrayrulewidth-2pt}|@{}}%
   \hline\framedhslinecorrect\\{-1.5ex}%
   \let\endoflinesave=\\
   \let\\=\@normalcr
   \(\pboxed}%
  {\endpboxed\)%
   \framedhslinecorrect\endoflinesave{.5ex}\hline
   \endtabular
   \parskip=\belowdisplayskip\par\noindent
   \ignorespacesafterend}

\newcommand{\framedhslinecorrect}[2]%
  {#1[#2]}

\newcommand{\framedhs}{\sethscode{framedhscode}}

% The inlinehscode environment is an experimental environment
% that can be used to typeset displayed code inline.

\newenvironment{inlinehscode}%
  {\(\def\column##1##2{}%
   \let\>\undefined\let\<\undefined\let\\\undefined
   \newcommand\>[1][]{}\newcommand\<[1][]{}\newcommand\\[1][]{}%
   \def\fromto##1##2##3{##3}%
   \def\nextline{}}{\) }%

\newcommand{\inlinehs}{\sethscode{inlinehscode}}

% The joincode environment is a separate environment that
% can be used to surround and thereby connect multiple code
% blocks.

\newenvironment{joincode}%
  {\let\orighscode=\hscode
   \let\origendhscode=\endhscode
   \def\endhscode{\def\hscode{\endgroup\def\@currenvir{hscode}\\}\begingroup}
   %\let\SaveRestoreHook=\empty
   %\let\ColumnHook=\empty
   %\let\resethooks=\empty
   \orighscode\def\hscode{\endgroup\def\@currenvir{hscode}}}%
  {\origendhscode
   \global\let\hscode=\orighscode
   \global\let\endhscode=\origendhscode}%

\makeatother
\EndFmtInput
%

\usetheme{CambridgeUS}

\usepackage{color}
\definecolor{syntax}{RGB}{0, 0, 0}
\definecolor{datatype}{RGB}{196, 6, 11}
\definecolor{class}{RGB}{168,37,39}
\definecolor{fieldname}{RGB}{0,0,162}
\definecolor{prelude}{RGB}{64,80,117}
\definecolor{numeral}{RGB}{0,0,205}
\definecolor{infixoperator}{RGB}{19, 19, 168}
\definecolor{constructor}{RGB}{196, 6, 11}
\definecolor{keyword}{RGB}{4, 58, 252}
\definecolor{special1}{RGB}{159,138,0}
\definecolor{string}{RGB}{3, 106, 7}
\definecolor{char}  {RGB}{3, 106, 7}

\newcommand{\lhsCHsyntax}[1]{\color{syntax}{{#1}}}
\newcommand{\lhsCHfunction}[1]{\color{infixoperator}{{#1}}}
\newcommand{\lhsCHinfixoperator}[1]{\color{infixoperator}{{#1}}}
\newcommand{\lhsCHprelude}[1]{\color{prelude}{\mathbf{#1}}}
\newcommand{\lhsCHkeyword}[1]{\color{keyword}{\textbf{#1}}}
\newcommand{\lhsCHconstructor}[1]{\color{constructor}{\textbf{#1}}}
\newcommand{\lhsCHtype}[1]{\color{datatype}{{#1}}}
\newcommand{\lhsCHclass}[1]{\color{class}{{#1}}}

\begin{document}


\title[Crisis]
{QuickCheck: A Lightweight Tool for Random Testing of Haskell Programs}
\subtitle{by Koen Claessen and John Hughes \\
          A discussion}
\author
{Naren Sundaravaradan}

\frame{\titlepage}

\begin{frame}
\frametitle{A better talk would be...}

... one that talked about how to avoid testing.

\end{frame}

\begin{frame}
\frametitle{Consider the identity function}

Say, that a client wants an identity function for integers in some API.

\pause

\begin{hscode}\SaveRestoreHook
\column{B}{@{}>{\hspre}l<{\hspost}@{}}%
\column{E}{@{}>{\hspre}l<{\hspost}@{}}%
\>[B]{} {\lhsCHfunction{id}}\mathbin{::} {\lhsCHtype{Int}}\to  {\lhsCHtype{Int}}{}\<[E]%
\\
\>[B]{} {\lhsCHfunction{id}}\;\Varid{x}\mathrel{=}\Varid{x}{}\<[E]%
\ColumnHook
\end{hscode}\resethooks

\pause

Unfortunately, this needs testing. Because the type allows for other
operations on integers and hence I could have also written a function
-- with the same signature -- that isn't the identity.

\begin{hscode}\SaveRestoreHook
\column{B}{@{}>{\hspre}l<{\hspost}@{}}%
\column{E}{@{}>{\hspre}l<{\hspost}@{}}%
\>[B]{} {\lhsCHfunction{id}}\mathbin{::} {\lhsCHtype{Int}}\to  {\lhsCHtype{Int}}{}\<[E]%
\\
\>[B]{} {\lhsCHfunction{id}}\mathrel{=}(\mathbin{+}\mathrm{1}){}\<[E]%
\ColumnHook
\end{hscode}\resethooks

\end{frame}



\begin{frame}
\frametitle{Still, this is better than C...}

... because in C -- if you want to -- you can delete everything in the filesystem in that function. So, purity helps. That signature in Haskell is pure and does not allow for side-effects.
\end{frame}

\begin{frame}
\frametitle{Can we do better?}

\pause

Consider the type

\begin{hscode}\SaveRestoreHook
\column{B}{@{}>{\hspre}l<{\hspost}@{}}%
\column{E}{@{}>{\hspre}l<{\hspost}@{}}%
\>[B]{} {\lhsCHfunction{id}}\mathbin{::}\Varid{a}\to \Varid{a}{}\<[E]%
\ColumnHook
\end{hscode}\resethooks

and think of the number of ways you can implement this...

\pause

\begin{hscode}\SaveRestoreHook
\column{B}{@{}>{\hspre}l<{\hspost}@{}}%
\column{E}{@{}>{\hspre}l<{\hspost}@{}}%
\>[B]{} {\lhsCHfunction{id}}\;\Varid{x}\mathrel{=}\Varid{x}{}\<[E]%
\ColumnHook
\end{hscode}\resethooks

We now have a function where the type signature \emph{uniquely}
determines the function implementation. There is only one
implementation and it is the identity function.

\pause

\textbf{Cough...} Haskell also defines a bottom value (so there are
two implementations). But, this is usually used for testing or raising
uncatchable exceptions.

\begin{hscode}\SaveRestoreHook
\column{B}{@{}>{\hspre}l<{\hspost}@{}}%
\column{E}{@{}>{\hspre}l<{\hspost}@{}}%
\>[B]{} {\lhsCHfunction{id}}\mathbin{::}\Varid{a}\to \Varid{a}{}\<[E]%
\\
\>[B]{} {\lhsCHfunction{id}}\;\anonymous \mathrel{=}\bot {}\<[E]%
\ColumnHook
\end{hscode}\resethooks

\end{frame}



\begin{frame}
\frametitle{Example (from Michael Snoyman's presentation)}

Say you are asked to design a method to read a list of employees from
a file where some employees don't have an ID. You have to assign ID's
for them and write the employees to another file. (Employee id's are
integers). \\

\pause

You might begin by writing an employee as...

\begin{hscode}\SaveRestoreHook
\column{B}{@{}>{\hspre}l<{\hspost}@{}}%
\column{E}{@{}>{\hspre}l<{\hspost}@{}}%
\>[B]{}\mathbf{data}\; {\lhsCHconstructor{Employee}}\mathrel{=} {\lhsCHconstructor{Employee}}\; {\lhsCHtype{String}}\; {\lhsCHtype{Int}}{}\<[E]%
\ColumnHook
\end{hscode}\resethooks

\pause

but, how will this let you represent employee's without id's without
resorting to C-like coding and writing $-1$ when there is no id? You
also don't want to expose the full power of integers like being able
to add, subtract, etc.

\end{frame}



\begin{frame}
\frametitle{Example (from Michael Snoyman's presentation)}

Use the type system and make your Employee type more general.

\begin{hscode}\SaveRestoreHook
\column{B}{@{}>{\hspre}l<{\hspost}@{}}%
\column{E}{@{}>{\hspre}l<{\hspost}@{}}%
\>[B]{}\mathbf{newtype}\; {\lhsCHconstructor{EmployeeId}}\mathrel{=} {\lhsCHconstructor{EmployeeId}}\; {\lhsCHtype{Int}}{}\<[E]%
\\
\>[B]{}\mathbf{data}\; {\lhsCHconstructor{Employee}}\;\Varid{a}\mathrel{=} {\lhsCHconstructor{Employee}}\; {\lhsCHtype{String}}\;\Varid{a}{}\<[E]%
\ColumnHook
\end{hscode}\resethooks

\pause

So, now when you read, assign, and write employees you can have these
nice signatures

\begin{hscode}\SaveRestoreHook
\column{B}{@{}>{\hspre}l<{\hspost}@{}}%
\column{5}{@{}>{\hspre}l<{\hspost}@{}}%
\column{13}{@{}>{\hspre}l<{\hspost}@{}}%
\column{E}{@{}>{\hspre}l<{\hspost}@{}}%
\>[B]{} {\lhsCHfunction{readEmployees}}\mathbin{::} {\lhsCHtype{FilePath}}\to  {\lhsCHtype{IO}}\;[\mskip1.5mu  {\lhsCHconstructor{Employee}}\;( {\lhsCHtype{Maybe}}\; {\lhsCHconstructor{EmployeeId}})\mskip1.5mu]{}\<[E]%
\\
\>[B]{} {\lhsCHfunction{assignID}}\mathbin{::} {\lhsCHconstructor{Employee}}\;( {\lhsCHtype{Maybe}}\; {\lhsCHconstructor{EmployeeId}})\to {}\<[E]%
\\
\>[B]{}\hsindent{13}{}\<[13]%
\>[13]{} {\lhsCHtype{IO}}\;( {\lhsCHconstructor{Employee}}\; {\lhsCHconstructor{EmployeeId}}){}\<[E]%
\\
\>[B]{} {\lhsCHfunction{writeEmployees}}\mathbin{::} {\lhsCHtype{FilePath}}\to [\mskip1.5mu  {\lhsCHconstructor{Employee}}\; {\lhsCHconstructor{EmployeeId}}\mskip1.5mu]\to  {\lhsCHtype{IO}}\;(){}\<[E]%
\\[\blanklineskip]%
\>[B]{} {\lhsCHfunction{readEmployees}}\;\Varid{inFile}{}\<[E]%
\\
\>[B]{}\hsindent{5}{}\<[5]%
\>[5]{} {\lhsCHinfixoperator{\ \mathbin{\ \bind\ }\ }} {\lhsCHfunction{mapM}}\;\Varid{assignId}{}\<[E]%
\\
\>[B]{}\hsindent{5}{}\<[5]%
\>[5]{} {\lhsCHinfixoperator{\ \mathbin{\ \bind\ }\ }} {\lhsCHfunction{writeEmployees}}\;\Varid{outFile}{}\<[E]%
\ColumnHook
\end{hscode}\resethooks

\end{frame}




\begin{frame}
\frametitle{Another example (modified from HaskellWiki)}

Say you have some data that you expose for the user to play with

\begin{hscode}\SaveRestoreHook
\column{B}{@{}>{\hspre}l<{\hspost}@{}}%
\column{E}{@{}>{\hspre}l<{\hspost}@{}}%
\>[B]{}\mathbf{data}\; {\lhsCHconstructor{SomeData}}\mathrel{=} {\lhsCHconstructor{SomeData}}\; {\lhsCHtype{String}}{}\<[E]%
\\[\blanklineskip]%
\>[B]{} {\lhsCHfunction{play}}\mathbin{::}( {\lhsCHtype{String}}\to  {\lhsCHtype{String}})\to  {\lhsCHconstructor{SomeData}}\to  {\lhsCHconstructor{SomeData}}{}\<[E]%
\\
\>[B]{} {\lhsCHfunction{play}}\; {\lhsCHfunction{f}}\;( {\lhsCHconstructor{SomeData}}\;\Varid{s})\mathrel{=} {\lhsCHconstructor{SomeData}}\;( {\lhsCHfunction{f}}\;\Varid{s}){}\<[E]%
\ColumnHook
\end{hscode}\resethooks

\pause

and you offer ways to eventually commit or display -- but you must ensure
that these things are done only after the data has been validated.

\begin{hscode}\SaveRestoreHook
\column{B}{@{}>{\hspre}l<{\hspost}@{}}%
\column{E}{@{}>{\hspre}l<{\hspost}@{}}%
\>[B]{} {\lhsCHfunction{validate}}\mathbin{::} {\lhsCHconstructor{SomeData}}\to  {\lhsCHtype{Bool}}{}\<[E]%
\\
\>[B]{} {\lhsCHfunction{save}}\mathbin{::} {\lhsCHconstructor{SomeData}}\to  {\lhsCHtype{IO}}\;(){}\<[E]%
\\
\>[B]{} {\lhsCHfunction{prettify}}\mathbin{::} {\lhsCHconstructor{SomeData}}\to  {\lhsCHtype{String}}{}\<[E]%
\ColumnHook
\end{hscode}\resethooks

\pause

Problem: you must either trust that the user validates before calling
\textbf{save} or \textbf{prettify} or you must invoke validate inside
those which will not only 1) break the rule of do one thing in a
function but 2) also makes the functions more partial because if the data
doesn't validate you must throw an exception or handle it some other
way.


\end{frame}



\begin{frame}
\frametitle{Once again, get the type system to help you}

This time we use phantom types.

\pause

\begin{hscode}\SaveRestoreHook
\column{B}{@{}>{\hspre}l<{\hspost}@{}}%
\column{E}{@{}>{\hspre}l<{\hspost}@{}}%
\>[B]{}\mathbf{data}\; {\lhsCHtype{Valid}}{}\<[E]%
\\
\>[B]{}\mathbf{data}\; {\lhsCHtype{Invalid}}{}\<[E]%
\\
\>[B]{}\mathbf{data}\; {\lhsCHconstructor{SomeData}}\;\Varid{a}\mathrel{=} {\lhsCHconstructor{SomeData}}\; {\lhsCHtype{String}}{}\<[E]%
\\[\blanklineskip]%
\>[B]{} {\lhsCHfunction{play}}\mathbin{::}( {\lhsCHtype{String}}\to  {\lhsCHtype{String}})\to  {\lhsCHconstructor{SomeData}}\; {\lhsCHtype{Invalid}}\to  {\lhsCHconstructor{SomeData}}\; {\lhsCHtype{Invalid}}{}\<[E]%
\\
\>[B]{} {\lhsCHfunction{play}}\; {\lhsCHfunction{f}}\;( {\lhsCHconstructor{SomeData}}\;\Varid{s})\mathrel{=} {\lhsCHconstructor{SomeData}}\;( {\lhsCHfunction{f}}\;\Varid{s}){}\<[E]%
\ColumnHook
\end{hscode}\resethooks

\pause

Now for the type-safe functions.

\begin{hscode}\SaveRestoreHook
\column{B}{@{}>{\hspre}l<{\hspost}@{}}%
\column{E}{@{}>{\hspre}l<{\hspost}@{}}%
\>[B]{} {\lhsCHfunction{validate}}\mathbin{::} {\lhsCHconstructor{SomeData}}\; {\lhsCHtype{Invalid}}\to  {\lhsCHtype{Maybe}}\;( {\lhsCHconstructor{SomeData}}\; {\lhsCHtype{Valid}}){}\<[E]%
\\
\>[B]{} {\lhsCHfunction{save}}\mathbin{::} {\lhsCHconstructor{SomeData}}\; {\lhsCHtype{Valid}}\to  {\lhsCHtype{IO}}\;(){}\<[E]%
\\
\>[B]{} {\lhsCHfunction{prettify}}\mathbin{::} {\lhsCHconstructor{SomeData}}\; {\lhsCHtype{Valid}}\to  {\lhsCHtype{String}}{}\<[E]%
\ColumnHook
\end{hscode}\resethooks

\end{frame}


\begin{frame}
\frametitle{Further examples can be found in...}

\begin{itemize}
\item \textbf{yesod} offers type-safe URLs, routing, etc
\item \textbf{persistent} offers type-safe serialization of data to/from DB
\item \textbf{esqueleto} type-safe SQL queries (joins and all)
\item \textbf{blaze-html} types ensure you never have unescaped entities (avoiding XSS)
\item ... and many more
\end{itemize}

\end{frame}



\begin{frame}
\frametitle{Back to testing}

QuickCheck is a Haskell package offering random testing of
user-defined properties. If allows one to

\begin{itemize}
\item define properties
\item define/derive generators for user-defined data
\item control/inspect distribution of random data
\end{itemize}

\end{frame}

\end{document}
